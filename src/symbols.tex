% Based on https://lospec.com/palette-list/na16
\definecolor{box-red}{HTML}{d26471}
\definecolor{box-blue}{HTML}{7ec4c1}
\definecolor{box-green}{HTML}{c0c741}
%
\definecolor{accent-blue}{HTML}{34859d}
\definecolor{accent-brown}{HTML}{d79b7d}
\definecolor{accent-gray}{HTML}{8c8fae}
\definecolor{accent-green}{HTML}{647d34}
\definecolor{accent-purple}{HTML}{70377f}
\definecolor{accent-red}{HTML}{9d303b}%

\tikzset{roll box/.style={inner ysep=0.5em, inner xsep=0.1em, fill=black!20}}
\tikzset{check roll box/.style={roll box, fill=box-blue!40}}
\tikzset{damage roll box/.style={roll box, fill=box-red!40, rounded corners=0.5em}}
\tikzset{healing roll box/.style={roll box, fill=box-green!40, rounded corners=0.5em}}
\tikzset{effect box/.style={roll box, fill=box-red!40, rounded corners=0.5em}}

\newcommand{\pfsymbol}[1]{%
  \mbox{\pfsymbols\selectfont#1}
}

% \Action[<trait 1>, <trait 2>, …]{<source book>}{<page>}{<actions>}[<action range>]{<name>}
%
% Use \Action*{…} to render action traits on a new line.
\NewDocumentCommand{\Action}{ommmom}{
  \subsection[#6]{#6\ActionSymbol{#4}\IfValueT{#5}{- \ActionSymbol{#5}} \IfValueT{#1}{\ActionTraits{#1}} \hfill \Reference{#2}{#3}}
}

% \Activity[<trait 1>, <trait 2>, …]{<source book>}{<page>}{<duration>}{<name>}
% 
% Use \Activity*{…} to render activity traits on a new line.
\NewDocumentCommand{\Activity}{sommmm}{
  \IfBooleanTF{#1}
    {\subsection[#6]{#6 {\small\engschrift (#5)} \hfill \Reference{#3}{#4}\\\ActionTraits{#2}}}
    {\subsection[#6]{#6 {\small\engschrift (#5)} \ActionTraits{#2} \hfill \Reference{#3}{#4}}}
}

% \Rule{<source book>}{<page>}{<name>}
\NewDocumentCommand{\Rule}{mmm}{
  \subsection[#3]{#3 \hfill \Reference{#1}{#2}}
}

\newcommand{\Requirements}[1]{\paragraph{Requirements} #1.}

\newcommand{\Attribute}[2]{\smallskip\mbox{\engschrift {\color{gray} #1} #2}}
\newcommand{\WeaponDamage}[1]{\Attribute{Damage}{#1}}
\newcommand{\ItemHands}[1]{\Attribute{Hands}{#1}}
\newcommand{\ItemBulk}[1]{\Attribute{Bulk}{#1}}
\newcommand{\ItemPrice}[1]{\Attribute{Price}{#1}}
\newcommand{\ItemUsage}[1]{\Attribute{Usage}{#1}}
\newcommand{\WeaponType}[1]{\Attribute{Type}{#1}}
\newcommand{\WeaponCategory}[1]{\Attribute{Category}{#1}}
\newcommand{\WeaponGroup}[1]{\Attribute{Group}{#1}}
\newcommand{\EffectDuration}[1]{\Attribute{Duration}{#1}}
\newcommand{\Frequency}[1]{\Attribute{Frequency}{#1}}
\newcommand{\Flavor}[1]{\textit{#1}}

\newcommand{\Hardness}[1]{\Attribute{Hardness}{#1}}
\newcommand{\BrokenThreshold}[1]{\Attribute{BT}{#1}}

% \HitPoints 
%   renders the Attribute for hit points (HP). 
% \HitPoints[<max>]
%   renders the Attribute for hit points (HP) and 
%   a box to track [current] / <max>.
\NewDocumentCommand{\HitPoints}{o}{%
  \Attribute{HP}{%
    \IfNoValueTF{#1}
      {}
      {\(\FormulaVariable{current}{\phantom{100}}\) / #1}}%
}

% \Reference{<source book>}{<page>}
\newcommand{\Reference}[2]{{\small \color{gray} \engschrift #1 #2}}

\newcommand{\CardTraits}[1]{\strut\raisebox{1em}[0pt][0pt]{\ActionTraits{#1}}}

\newcommand{\ActionTraits}[1]{{%
  \small \engschrift
  \foreach[count=\index] \trait in {#1}{\ifnum \index > 1 ~~ \fi \DamageType{\trait}[]\trait}%
}}

% TODO fix distance between actions, maybe avoid draw in favor of more fill?
\newcommand{\ActionSymbol}[1]{
  \ifnum #1 = -1
    \pfsymbol{reaction}
  \fi
  \ifnum #1 = 0
    \pfsymbol{free-action}
  \fi
  \ifnum #1 = 1
    \pfsymbol{one-action}
  \fi
  \ifnum #1 = 2
    \pfsymbol{two-actions}
  \fi
  \ifnum #1 = 3
    \pfsymbol{three-actions} 
  \fi           
}

% \FormulaVariable[<mod alignment>]{<mod desc>}{<content>}
\newcommand{\FormulaVariable}[3][c]{
  \operatorname{
    \tikz[baseline, trim right=(box.east)] \path
      node[draw, shape=rectangle, anchor=base, fill=white] (box) { #3 }
      \ifnum\pdfstrcmp{#1}{c}=0
      (box.south) node[below=-0.5ex] {\makebox[0pt][#1]{\scriptsize \engschrift #2}}
      \fi
      \ifnum\pdfstrcmp{#1}{r}=0
      (box.south east) +(0.5em, 0em) node[below=-0.5ex] {\makebox[0pt][#1]{\scriptsize \engschrift #2}}
      \fi
      ;
    }
  }

% \CheckFormula[<mod alignment>]{<mod desc>}[<DC>]
\NewDocumentCommand{\CheckFormula}{O{c}mo}{
  \(
    \operatorname{1d20} + \FormulaVariable[#1]{
      \foreach[count=\index] \summand in {#2}
        {\ifnum \index > 1 ~+~ \fi \summand}
      }{\phantom{100}}\IfValueT{#3}{\ge \operatorname{\mbox{#3}}}%
  \)%
}

% \CheckRoll{<check type>}[<mod alignment>]{<mod desc>}[<DC>]
\NewDocumentCommand{\CheckRoll}{mO{c}mo}{
  \tikz[baseline=(content.base)]{
    \node (content)  {%
      \footnotesize
      \strut\smash{%
        % \parbox[t]{1.25em}{\CheckType{#1}[]}
        \CheckFormula[#2]{#3}[#4]%
      }%
    };
    \begin{pgfonlayer}{background}
      \node [fit=(content), check roll box] (box) {};
    \end{pgfonlayer}%
  }%
}

% \DamageRoll{<damage type>}[<alternative damage type>]{<damage roll>}{<mod descr>}
\NewDocumentCommand{\DamageRoll}{momm}{
  \tikz[baseline=(content.base)]{
    \node (content) {%
      \footnotesize
      \strut\smash{
        \IfValueTF{#2}
          {%
            \raisebox{0.4em}{\DamageType{#1}}%
            \clap{\tikz[baseline] \draw (-0.6em, -0.4em) -- (0.6em, 1.2em);}%
            \raisebox{-0.4em}{\DamageType{#2}}%
          }
          {\ifnum\pdfstrcmp{#1}{}=0 \else\DamageType{#1}\fi}
        \(
          \mbox{#3}
          \foreach[count=\index] \part in {#4} {
            \pgfmathparse{\index > 1 || \pdfstrcmp{#3}{} != 0}
            \ifnum\pgfmathresult=1 + \fi
            \FormulaVariable{\part}{\phantom{100}}
          }
        \)
      }%
    };
    \begin{pgfonlayer}{background}
      \node [fit=(content), damage roll box] (box) {};
    \end{pgfonlayer}%
  }%
}

% \HealingRoll{<healing roll>}{<mod descr>}
\NewDocumentCommand{\HealingRoll}{mm}{
  \tikz[baseline=(content.base)]{
    \node (content) {%
    \footnotesize
      \strut\smash{
        \(
          \mbox{#1}
          \foreach[count=\index] \part in {#2} {
            \pgfmathparse{\index > 1 || \pdfstrcmp{#1}{} != 0}
            \ifnum\pgfmathresult=1 + \fi
            \FormulaVariable{\part}{\phantom{1}}
          }
        \)
      }
    };
    \begin{pgfonlayer}{background}
      \node [fit=(content), healing roll box] (box) {};
    \end{pgfonlayer}
  }
}

% \Effect{success | failure}{<description>}   - for regular success or failure
% \Effect*{success | failure}{<description>}  - for critical success or critical failure
\NewDocumentCommand{\Effect}{smm}{
  \pfsymbol{\IfBooleanT{#1}{critical-}#2} & #3 \\
}

% \EffectBox{%
%   \Effect*{success}{…}
%   \Effect{success}{…}
%   \Effect{failure}{…}
%   \Effect*{failure}{…}
% }
% TODO: adjust colored box based on number of rows in the table
\NewDocumentCommand{\EffectBox}{m}{
  \tikz[baseline=(content.base)]{
    \node (content) {\strut\smash{
      {\setlength{\tabcolsep}{0pt}\begin{tabular}{ll}#1\end{tabular}}
    }};
    \begin{pgfonlayer}{background}
      \node [fit=(content), effect box] (box) {};
    \end{pgfonlayer}
  }
}

\NewDocumentCommand{\CheckType}{mo}{
  \ifcsname CheckType#1\endcsname
  \csname CheckType#1\endcsname
  \else
  \IfValueTF{#2}{#2}{#1}
  \fi
}

\NewDocumentCommand{\DamageType}{mo}{%
  \ifcsname DamageType#1\endcsname
  \csname DamageType#1\endcsname
  \else
  \IfValueTF{#2}{#2}{#1}
  \fi
}

% Keep this for future reference when importing icons from game-icons.net
\tikzset{game-icons.net/.style={fill,scale=0.002em,yscale=-1}}

\newcommand{\DamageTypeFire}{\pfsymbol{fire}}
\newcommand{\DamageTypeCold}{\pfsymbol{cold}}
\newcommand{\DamageTypeSlashing}{\pfsymbol{slashing}}
\newcommand{\DamageTypeBludgeoning}{\pfsymbol{bludgeoning}}
\newcommand{\DamageTypePiercing}{\pfsymbol{piercing}}
\newcommand{\DamageTypeBleed}{\pfsymbol{bleeding}}
\newcommand{\DamageTypeMental}{\pfsymbol{mental}}
\newcommand{\DamageTypeForce}{\pfsymbol{force}}
\newcommand{\DamageTypeAcid}{\pfsymbol{acid}}
\newcommand{\DamageTypeElectricity}{\pfsymbol{electricity}}
\newcommand{\DamageTypeMagic}{\pfsymbol{magical}}
\newcommand{\DamageTypeSonic}{\pfsymbol{sonic}}
\newcommand{\DamageTypePoison}{\pfsymbol{poison}}
\newcommand{\DamageTypeGood}{\pfsymbol{good}}
\newcommand{\DamageTypeEvil}{\pfsymbol{evil}}
\newcommand{\DamageTypeChaotic}{\pfsymbol{chaotic}}
\newcommand{\DamageTypeLawful}{\pfsymbol{lawful}}
\newcommand{\DamageTypeNegative}{\pfsymbol{negative}}
\newcommand{\DamageTypePositive}{\pfsymbol{positive}}
